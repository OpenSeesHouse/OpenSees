% File: ~/domain/domain/partitioned/PartitionedDomain.tex 
% What: "@(#) PartitionedDomain.tex, revA"

UNDER CONSTRUCTION

\noindent {\bf Files}   \\
\indent \#include $<$/domain/domain/partitioned/PartitionedDomain.h$>$  \\

\noindent {\bf Class Decleration}  \\
\indent class PartitionedDomain: public Domain  \\

\noindent {\bf Class Hierarchy} \\
\indent  Domain \\
\indent\indent {\bf PartitionedDomain} \\


\noindent {\bf Description}  \\
\indent PartitionedDomain is an extension of Domain. A partitioned
domain is an aggregation of subdomains. All elements, nodes, loadcases
are added to the PartitionedDomain. The components can be moved among 
subdomains (keeping in mind that subdomains are themselves subclasses
of domain and therefore must obey the requirements for adding and
removing elements specified in the interface for Domain) by invoking
the {\em remove..()} and {\em add...()} methods on the subdomain. \\


\noindent {\bf Class Interface}  \\
\indent\indent // Constructors  \\
\indent\indent {\em PartitionedDomain(DomainPartitioner \&thePartitioner); } \\
\indent\indent {\em PartitionedDomain(int numNodes, int numElements, int
numLCs, int numSPs, int numMPs, \\ 
\indent\indent\indent\indent int numSubdomains, DomainPartitioner
\&thePartitioner); } \\  \\
\indent\indent //  Destructor  \\
\indent\indent {\em virtual~ $\tilde{}$PartitionedDomain();}  \\ \\
\indent\indent // Public Member Functions - which extend the Domain class \\
\indent\indent {\em virtual int partition(int numPartitions);}\\
\indent\indent {\em virtual bool addSubdomain(Subdomain *theSubdomainPtr);}  \\
\indent\indent {\em virtual int getNumSubdomains(void);}\\
\indent\indent {\em virtual Subdomain *getSubdomainPtr(int tag);}\\
\indent\indent {\em virtual SubdomainIter \&getSubdomains(void);} \\
\indent\indent {\em Node *removeExternalNode(int tag); } \\
\indent\indent {\em Graph \&getSubdomainGraph(void); } \\ \\
\indent\indent // Public Member Functions - inherited from Domain but
rewritten \\
\indent\indent {\em virtual bool addElement(Element *theElementPtr, bool
check = false);}  \\
\indent\indent {\em virtual bool addNode(Node *theNodePtr, bool check = false);}  \\
\indent\indent {\em virtual bool addSP\_Constraint(SP\_Constraint *theSPptr, bool
check = false);}  \\
\indent\indent {\em virtual bool addMP(MP\_Constraint *theMPptr, bool check = false);}  \\
\indent\indent {\em virtual bool addLoadCase(LoadCase *theLCptr);}  \\
\indent\indent {\em virtual ElementIter \&getElements(void);} \\
\indent\indent {\em virtual  Element *getElement(int tag) const;}\\
\indent\indent {\em virtual  Node *getNode(int tag) const;}\\
\indent\indent {\em virtual  LoadCase *getLoadCasePtr(int tag) const;}\\
\indent\indent {\em virtual bool setCurrentLoadCase(int LCtag);} \\
\indent\indent {\em virtual void applyLoad(double time = 0.0,
double loadFactor = 1.0);} \\
\indent\indent {\em virtual void linearize(void);} \\
\indent\indent {\em virtual void commit(void);} \\
\indent\indent {\em virtual int getCurrentLoadCase(void);} \\
\indent\indent {\em virtual int getNumElements(void) const;}\\
\indent\indent {\em virtual int getNumNodes(void) const = 0;}\\
\indent\indent {\em virtual int getNumSPs(void) const;}\\
\indent\indent {\em virtual int getNumMPs(void) const;}\\
\indent\indent {\em virtual int getNumLCs(void) const;}\\    
\indent\indent {\em virtual Domain *getEmptyDomainCopy(void);}\\
\indent\indent {\em virtual Element *removeElement(int tag);}\\
\indent\indent {\em virtual Node *removeNode(int tag, bool checkNeeded
= true);}\\ 
\indent\indent {\em virtual LoadCase *removeLoadCase(int tag);}\\        
\indent\indent {\em virtual SP\_Constraint *removeSP\_Constraint(int
tag);}\\ 
\indent\indent {\em virtual MP\_Constraint *removeMP\_Constraint(int
tag);} \\ \\
\indent\indent // Protected Methods   \\
\indent {\em DomainPartitioner \&getPartitioner(void) const;} \\
{\em virtual int buildEleGraph(Graph \&theEleGraph)} \\



\noindent {\bf Constructors}  \\
\indent {\em PartitionedDomain(DomainPartitioner \&thePartitioner); } \\
Constructs an empty PartitionedDomain. A link with the domain
partitioner {\em thePartitioner} is set up. The {\em thePartitioner}
is used by the domain to partition and load balance the partitioned domain. \\

\indent {\em PartitionedDomain(int numNodes, int numElements, int
numLCs, int numSPs, int numMPs, \\ 
\indent\indent\indent int numSubdomains, DomainPartitioner
\&thePartitioner); } \\ 
Constructs an empty PartitionedDomain, storage is allocated for the components
that are to be added using the estimated number of components passed
as arguments. A link with the domain partitioner {\em thePartitioner}
is set up. The {\em thePartitioner} is used by the domain to partition
and load balance the partitioned domain. \\


\noindent {\bf Destructor}  \\
\indent {\em virtual~ $\tilde{}$PartitionedDomain();}  \\
Deletes the storage components. \\

\noindent {\bf Public Member Functions - which extend the Domain class}  \\
\indent {\em virtual int partition(int numPartitions);}\\
Method which first checks that subdomains with tags 1 through {\em numPartitions} exist in the 
PartitionedDomain. Then it invokes {\em setPartitionedDomain(*this)} on the DomainPartitioner
and finally it returns the result of invoking {\em partition(numPartitions} on
the DomainPartitioner, which will return 0 if succesfull, a negative number if not. \\

{\em virtual bool addSubdomain(Subdomain *theSubdomainPtr);}  \\
Adds the subdomain pointed to by theSubdomainPtr to the domain. The domain
is responsible for checking that no other subdomain with a similar tag,
has been previously added to the domain. If successfull
the domain is responsible for invoking {\em setDomain(this)} on the
Subdomain. The domain is also responsible for invoking {\em
domainChange()}. The call returns {\em false} if the subdomain was not added, 
otherwise {\em true} is returned. \\  

{\em virtual int getNumSubdomains(void);}\\
Method which returns the number of Subdomains (partitions). \\

{\em virtual Subdomain *getSubdomainPtr(int tag);}\\
Returns the Subdomain whose tag is given by {\em tag}. \\

{\em virtual SubdomainIter \&getSubdomains(void);} \\
Returns an iter for the Subdomains of the PartitionedDomain. \\


{\em Node *removeExternalNode(int tag); } \\
A method to remove a Node whose tag is given by {\em tag} from the PartitionedDomain, 
but will not remove the Node from any Subdomains. \\

{\em Graph \&getSubdomainGraph(void); } \\
This will create a new graph each time it is invoked; deleting the old graph. THIS WILL
BE CHANGED.  A vertex is created for each Subdomain, with an edge to each Subdomain the 
Subdomain is connected to, a tag equal to the Subdomain tag,  and a weight equal to the 
result of invoking {\em getCost()} on the Subdomain. \\

\noindent {\bf Public Member Functions - inherited from Domain but rewritten}\\
\indent {\em virtual bool addElement(Element *theElementPtr, bool
check = false);}  \\
To add the element pointed to by theElementPtr to the domain. If {\em check}
is {\em true} the domain is responsible for checking to see that: 1)
no other element with a similar tag, element number, exists in any of
the subdomains. If check is successfull the partitioned domain
attempts to add the element to the storage arrey. The call returns
{\em false} if the element was not added, otherwise {\em true} is 
returned.\\   

{\em virtual bool addNode(Node *theNodePtr, bool check = false);}  \\
Adds the node pointed to by theNodePtr to the domain. If {\em check}
is {\em true} the domain is responsible for checking that no other
node with a similar tag, node number, exists in any of the
subdomains. If successfull the partition domain attempts to add the
node by invoking {\em Domain::addNode}. The call returns {\em false} if
the node was not added, otherwise {\em true} is returned. \\  


{\em virtual bool addSP\_Constraint(SP\_Constraint *theSPptr, bool
check = false);}  \\
Adds the single point constraint pointed to by theSPptr to the 
domain. The domain performs some checks is {\em check} is true. If
successful the domain adds the constraint using {\em
Domain::addSP\_Constraint()}. The call returns {\em false} if
the constraint was not added, otherwise {\em true} is returned. \\  

{\em virtual bool addMP(MP\_Constraint *theMPptr, bool check = false);}  \\
Adds the multiple point constraint pointed to by theMPptr, to the
domain. The domain performs some checks is {\em check} is true. If
successful the domain adds the constraint using {\em
Domain::addMP\_Constraint()}. The call returns {\em false} if
the constraint was not added, otherwise {\em true} is returned. \\  


{\em virtual bool addLoadCase(LoadCase *theLCptr);}  \\
\\

%%%%%%%% Public Member Functions - ACCESS
\noindent {\bf Pure Virtual Public Member Functions - Access}  \\
\indent {\em virtual ElementIter \&getElements(void);} \\
It returns an {\em PartitionedDomEleIter} for the elements of the domain. This
is an iter which goes through all the subdomains, invoking {\em
getElements()} on the subdomain to get an ElementIter. The
PartitionedDomEleIter uses this iter to go through the elements of the
subdomain until it begins returning $0$; at which point it goes on to
the next subdomain. \\

{\em virtual  Element *getElement(int tag) const;}\\
Returns a pointer to the element whose tag is given by {\em tag}. If
no such element exists $0$ is returned. This is done by invoking {\em
getElement(tag)} on the subdomains until the element is found or no
more subdomains exist; in which case a $0$ is returned. \\

{\em virtual  Node *getNode(int tag) const;}\\
Returns a pointer to the node whose tag is given by {\em tag}. If
no such node exists $0$ is returned. This is done by invoking {\em
getNode(tag)} on the subdomains until the element is found or no
more subdomains exist; in which case a $0$ is returned. \\

{\em virtual  LoadCase *getLoadCasePtr(int tag) const;}\\
Returns a pointer to the element whose tag is given by {\em tag}. If
no such load case exists $0$ is returned. \\

\indent {\em virtual bool setCurrentLoadCase(int LCtag);} \\
Sets the current load case of the domain to be that whose tag is given
by LCtag. It iterates through all the subdomains invoking the same
operation on them. Returns {\em false} if no such load case exists, otherwise
returns {\em true}. \\

\indent {\em virtual void setCurrentTime(double newTime);} \\
Sets the current load case of the domain to be that whose tag is given
by {\em newTime}. It iterates through all the subdomains invoking the
same operation on them. \\

{\em virtual void applyLoad(double time = 0.0, double loadFactor
= 1.0);} \\
The partioned domain iterates through all the subdomains invoking {\em
applyLoad(double timeStamp)} on them. \\

{\em virtual void linearize(void);} \\
The partioned domain iterates through all the subdomains invoking {\em
linearize()} on them. \\


{\em virtual void commit(void);} \\
The partioned domain iterates through all the subdomains invoking {\em
commit()} on them. \\

%%%%%%%% Public Member Functions - QUERY
\noindent {\bf Public Member Functions - Query}  \\
\indent {\em virtual int getCurrentLoadCase(void);} \\
Returns the tag of the current load case set for the domain. If no
load case is set $-1$ is returned. \\

\indent {\em virtual double getCurrentTime(void);} \\
Returns the currentTime set for the domain. If no load case is set $0$
is returned. \\ 

\noindent {\bf Pure Virtual Public Member Functions - Query}  \\
\indent {\em virtual int getNumElements(void) const;}\\
Returns the number of elements in the domain. This number is obtained
by summing the contributions from each subdomain. \\

{\em virtual int getNumNodes(void) const = 0;}\\
Returns the number of nodes in the domain.
This number is obtained by summing the contributions from each subdomain. \\

{\em virtual int getNumSPs(void) const;}\\
Returns the number of SP\_Constraints in the domain. 
This number is obtained by summing the contributions from each subdomain. \\

{\em virtual int getNumMPs(void) const;}\\
Returns the number of MP\_Constraints in the domain. 
This number is obtained by summing the contributions from each subdomain. \\

{\em virtual int getNumLCs(void) const;}\\    
Returns the number of LoadCases in the domain. \
This number is obtained by summing the contributions from each subdomain. \\

{\em virtual Domain *getEmptyDomainCopy(void);}\\
Returns an empty copy of the actual domain.\\

{\em virtual Element *removeElement(int tag);}\\
To remove the element whose tag is given by {\em tag} from the
domain. The method Returns $0$ if no such element exists in the domain. Otherwise 
the domain invokes {\em setDomain(0)} on the element and {\em
setDomainChange(true,true,false)} on itself before a pointer to the
element is returned. \\

{\em virtual Node *removeNode(int tag, bool checkNeeded = true);}\\    
To remove the node whose tag is given by {\em tag} from the domain. 
Returns $0$ if no such node exists in the domain. Otherwise
if the {\em checkNeeded} is {\em true} before the node is removed a
check is made to see that the node is not referenced by any element,
constraint or load. If it is referenced the Node will not be
removed and $0$ is returned. If the node is to be removed the domain
invokes {\em setDomain(0)} on the node and {\em
setDomainChange(true,false,true)} on itself before a pointer to the
Node is returned.\\

{\em virtual LoadCase *removeLoadCase(int tag);}\\        
To remove the load case whose tag is given by {\em tag} from the domain. 
Returns $0$ if the load case was not in the domain, otherwise
returns a pointer to the load case that was removed. Invokes {\em
setDomain(0)} on the load case before it is returned.\\

{\em virtual SP\_Constraint *removeSP\_Constraint(int tag);}\\
To remove the SP\_Constraint whose tag is given by {\em tag} from the
domain. Returns $0$ if the constraint was not in the domain, otherwise
the domain invokes {\em setDomain(0)} on the constraint and {\em
setDomainChange(true,false,false)} on itself before a pointer to the
constraint is returned.\\  

{\em virtual MP\_Constraint *removeMP\_Constraint(int tag);} \\
To remove the MP\_Constraint whose tag is given by {\em tag} from the
domain. Returns $0$ if the constraint was not in the domain, otherwise
the domain invokes {\em setDomain(0)} on the constraint and {\em
setDomainChange(true,false,false)} on itself before a pointer to the
constraint is returned.\\  




\noindent {\bf Protected Member Functions}  \\
\indent {\em DomainPartitioner \&getPartitioner(void) const;} \\
Will return a pointer to the DomainPartitioner object associated with the
PartitionedDomain. \\

{\em virtual int buildEleGraph(Graph \&theEleGraph)} \\
A method which will cause the domain to discard the current element
graph and build a new one based on the element connectivity. Returns
$0$ if successfull otherwise $-1$ is returned along with an error
message to opserr. \\
