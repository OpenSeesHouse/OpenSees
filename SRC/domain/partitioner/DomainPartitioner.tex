% File: ~/domain/domain/partitioner/DomainPartitioner.tex 
% What: "@(#) DomainPartitioner.tex, revA"

UNDER CONSTRUCTION

\noindent {\bf Files}   \\
\indent \#include $<\tilde{ }$/domain/partitioner/DomainPartitioner.h$>$  \\

\noindent {\bf Class Decleration}  \\
\indent class DomainPartitioner \\

\noindent {\bf Class Hierarchy} \\
\indent {\bf DomainPartitioner} \\


\noindent {\bf Description}  \\
\indent A DomainPartitioner is an object used to partition and load balance
a PartitionedDomain. The DomainPartitioner uses the element graph of
the domain to partition and load balance. Derived types can use the
node graph of the domain. The partitioner uses a GraphPartitioner and
a LoadBalancingAlgo to partition and load balance the domain. \\

\noindent {\bf Class Interface}  \\
\indent  // Constructors  \\
\indent {\em DomainPartitioner(GraphPartitioner \&theGraphPartitioner,\\
\indent\indent\indent	LoadBalancer \&theLoadBalancer);} \\
\indent {\em DomainPartitioner(GraphPartitioner
\&theGraphPartitioner);}\\ \\
\indent Destructor  \\
\indent {\em virtual~ $\tilde{}$DomainPartitioner();}  \\ \\
\indent // Public Methods  \\
\indent{\em virtual void setPartitionedDomain(PartitionedDomain
\&theDomain);}\\
\indent {\em virtual int partition(int numParts);}\\
\indent {\em virtual void balance(Graph \&theWeightedGraph)} \\ \\
\indent // Public Methods Used by the LoadBalancer  \\
\indent {\em virtual Graph \&getNumPartitions(void)} \\
\indent {\em virtual Graph \&getPartitionGraph(void)} \\
\indent {\em virtual Graph \&getColoredGraph(void); } \\
\indent {\em virtual void swapVertex(int from, int to, int vertexTag,
bool notAdjacentOther = true); } \\ 
\indent {\em virtual void swapBoundary(int from, int to, bool
notAdjacentOther = true); } \\ 
\indent {\em virtual int releaseVertex(int from, \\
\indent\indent\indent\indent\indent\indent int vertexTag, \\
\indent\indent\indent\indent\indent\indent Graph \&theWeightedPartitionGraph, \\
\indent\indent\indent\indent\indent\indent double factorGreater, \\
\indent\indent\indent\indent\indent\indent bool adjacentVertexOonly); } \\
\indent {\em virtual int releaseBoundary(int from, \\
\indent\indent\indent\indent\indent\indent Graph \&theWeightedPartitionGraph, \\
\indent\indent\indent\indent\indent\indent double factorGreater, \\
\indent\indent\indent\indent\indent\indent bool adjacentVertexOonly); } \\



\noindent {\bf Constructors}  \\
\indent {\em DomainPartitioner(GraphPartitioner \&theGraphPartitioner,\\
\indent\indent\indent	LoadBalancer \&theLoadBalancer);} \\
Constructs a DomainPartitioner which will use {\em
theGraphPartitioner} to initially partition the PartitionedDomain
using the element graph and the {\em theLoadBalancer} to load balance 
the PartitionedDomain. The max number of subdomains that the Domain can be 
partitioned is currently set at 8.\\

\indent {\em DomainPartitioner(GraphPartitioner \&theGraphPartitioner);}\\
Constructs a DomainPartitioner which will use {\em
theGraphPartitioner} to initially partition the PartitionedDomain. The
max number of subdomains that the Domain can be partitioned is
currently set at 8.\\ 

\noindent {\bf Destructor}  \\
\indent {\em virtual~ $\tilde{}$DomainPartitioner();}  \\


\noindent {\bf Public Member Functions}  \\
\indent{\em virtual void setPartitionedDomain(PartitionedDomain
\&theDomain);}\\
Sets the link with the PartitionedDomain that is to be partitioned. \\

{\em virtual int partition(int numParts);}\\
Method invoked to partition the Domain. It first checks to see that
the PartitionedDomain has at least {\em numParts} Subdomains, with tags
1 through {\em numParts}; if not prints an error message and returns -1. 
It then asks the domain for the element graph. This graph is then partitioned 
using the GraphPartitioner into {\em numParts}; if partitioning fails an error 
message is printed and  $-10 +$ number returned from GraphPartitioner is
returned. If successfull the domain is partitioned according to the
following rules: \begin{itemize}
\item All nodes which are internal to a partition are added using the
{\em addNode()} method of the Subdomain. These nodes are removed from
the PartitionedDomain using {\em removeNode()}. 
\item External nodes (these are nodes shared across partitions as a
result of element connectivity or MP\_Constraints are added to those
Subdomains whose elements reference them. They are added using the
{\em addExternalNode()} command. 
\item SP\_Constraints whose node is interior to a Subdomain are removed
from the PartitionedDomain and added to the Subdomain. 
\item MP\_Constraints whose two nodes are interior to a Subdomain are
removed from the PartitionedDomain and added to the Subdomain.
\item The elements are sent to the partition whose tag is given by the
color of the vertex in the partitioned (colored) element graph. The
elements are removed from the PartitionedDomain using {\em
removeElement()} and added to the Subdomain using {\em addElement()}.
\item For the loads, a check is made to ensure that each Subdomain has
a LoadCase with a tag equal to the tags in the LoadCases that have
been added to the PartitionedDomain; if not new LoadCases are created
and added to the Subdomain. It then iterates through all the
NodalLoads in the LoadCases in the PartitionedDomain, if the
corresponding node is external the NodalLoad is removed and added to
the corresponding LoadCase in the Subdomain. ELEMENTAL LOADS are not
yet dealt with. 
\end{itemize}

The DomainPartitioner invokes {\em hasDomainChanged()} on each Subdomain; if the Subdomain 
has changed {\em invokeChangeOnAnalysis()}. Finally {\em
hasDomainChanged()} is invoked on the PartitionedDomain; if it has
changed {\em invokeChangeOnAnalysis()}. {\em partitionFlag} is set to
true. \\ 


{\em virtual void balance(Graph \&theWeightedGraph)} \\
Checks first to see that the {\em partitionFlag} has been set; if it
hasn'nt an error message is printed and a $-1$ is returned. If a
LoadBalancer was passed in the constructor {\em balance()} is invoked
on this object; if no LoadBalancer was passed nothing is done and
method returns$0$. If balancing is performed, the DomainPartitioner
invokes {\em hasDomainChanged()} on each Subdomain; if the Subdomain
has changed {\em invokeChangeOnAnalysis()}. Finally {\em
hasDomainChanged()} is invoked on the PartitionedDomain; if it has
changed {\em invokeChangeOnAnalysis()}. {\em partitionFlag} is set to
true. \\ 

Method which invokes {\em setPartitioner(this)} on the
LoadBalancingAlgo. It then invokes {\em balance(load)} on this
object, where {\em load} is vector of size {\em numParts}
containing the load of each subdomain. \\ 

\noindent {\bf Public Member Functions Used by the Balancing Algorithm}  \\
\indent {\em virtual Graph \&getNumPartitions(void)} \\
Returns the number of partitions in the PartitionedDomain. \\

\indent {\em virtual Graph \&getPartitionGraph(void)} \\
Method which returns the partition graph. This is a graph with a
vertex for every partition and an edge between partitions if there
exists an element in one partition which is connected to an element
in the other partition. \\

{\em virtual Graph \&getColoredGraph(void); } \\
A method which returns the current colored graph representing the
partitioning of the elements in the subdomains. Does this by invoking
{\em getElementGraph()} on the PartitionedDomain. Note that this is
the same graph that was colored by the DomainPartitioner in
partitioning the PartitionedDomain. \\ 

{\em virtual void swapVertex(int from, int to, int vertexTag, bool
notAdjacentOther = true); } \\ 
Method which will take the element given by vertex reference of the
vertex whose tag is given by {\em vertexTag} from subdomain {\em from}
and place it in subdomain {\em to}. If {\em notAdjacentOther} is {\em
true} a check is made to ensure that the vertex to be swapped is not
adjacent to a vertex in any other partition. Returns $0$ if
successfull, returns an error message and $-1$ if PartitionedDomain
has not been partitioned, $-2$ if {\em from} Subdomain does not exist,
$-3$ if {\em to Subdomain} does not exist, $-4$ if a vertex given by
{\em tag} does not exist, returns $-5$ if {\em notAdjacentOther} was
true and the vertex was adjacent to a vertex in another partition, and
returns $-6$ if no Element with a tag similar to {\em tag} exists
(this should not happen if element graph is built correctly). \\
The Element, Nodes, NodalLoads, SP\_Constraints and MP\_Constraints
that need to be moved between the two Subdomains, or between the
PartitionedDomain and Subdomains are also moved. NO ELEMENTAL LOADS 
are moved yet. \\


{\em virtual void swapBoundary(int from, int to, bool notAdjacentOther
= true); } \\ 
Method which when invoked will take all the boundary elements in
subdomain {\em from} that are connected to elements in subdomain {\em
to} and place them in subdomain {\em to}. If {\em adjacentVertexOther} is
{\em true} no Elements which are connected to elements in subdomains other 
than {\em to} and {\em from} are moved. Returns $0$ if successfull, returns 
an error message and $-1$ if PartitionedDomain has not been partitioned, $-2$ 
if {\em from} Subdomain does not exist, $-3$ if {\em to Subdomain}
does not exist. \\ The Elements, Nodes, NodalLoads, SP\_Constraints
and MP\_Constraints that need to be moved between the two Subdomains,
or between the PartitionedDomain and Subdomains are also moved. NO
ELEMENTAL LOADS are moved yet. It performs the operation by creating
an ID of vertices and then using code similar to that used in {\em
swapVertex()}; {\em swapVErtex()} is not called repeatedly as this was
found to be too slow. \\ 

{\em virtual int releaseVertex(int from, \\
\indent\indent\indent\indent\indent int vertexTag, \\
\indent\indent\indent\indent\indent Graph \&theWeightedPartitionGraph, \\
\indent\indent\indent\indent\indent double factorGreater, \\
\indent\indent\indent\indent\indent bool adjacentVertexOonly); } \\
Method which when invoked will take the element given by vertex
reference of the vertex whose tag is given by {\em vertexTag} from
subdomain {\em from} and place it in the subdomain to which it is most
attracted (to which it is most connected). If it is equally attracted
to two subdomains it is sent to the one with the lightest load (the
loads on the subdomains are given in the {\em
theWeightedPartitionGraph}. If the {\em mustReleaseToLighter} is {\em
true} a check is first made to see if the load on the intended
subdomain is lighter than the load in {\em from} and that the ratio in
load between from and the new domain is greater than {\em
factorGreater}; if this is the case the element is swapped, if not the
element is not swapped. An additional requirement that the vertex that
is to be swapped is not adjacent to any other partition can also be
set. \\ 
The method determines which partition the vertex is to be sent to and
then returns the result of invoking {\em swapVertex()} on itself, with
this partition used as the {\em to} argument in the arguments. \\

{\em virtual int releaseBoundary(int from, \\
\indent\indent\indent\indent\indent Graph \&theWeightedPartitionGraph, \\
\indent\indent\indent\indent\indent double factorGreater, \\
\indent\indent\indent\indent\indent bool adjacentVertexOonly); } \\
Method which when invoked will release all the elements on the
boundary of subdomain {\em from}. It performs the operation by
creating an ID of all the vertices on the boundary of the {\em from}
Subdomain. Then {\em releaseBoundary()} is invoked on all these vertices. \\


